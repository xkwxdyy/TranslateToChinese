\chapter{关于这个参考指南}

此参考指南旨在为初、中、高级的Pixinsight用户提供帮助。这意味着此参考指南在技术方面不得不上升到一个进阶的高度,所以对那些天体图像处理还处于新手状态的用户来说读起来可能会有一些困难。对于这些用户而言,完全手册可能更有帮助,在那里可以学到工作流程,处理技术和许多用实际数据说明的实例,并且书中对它们的讲解节奏适中,容易跟上。

此参考指南不能做什么:

\begin{itemize}
  \item 它不会教你怎么去处理图像。
  \item 它不会展示具体的例子或者指导。
  \item 它并不会解释Pixinsight的用户界面如何工作。
\end{itemize}

如果想找到关于以上几点或者更多的讲解,请参考完全手册。

此参考指南会做什么:

\begin{itemize}
  \item 提供对Pixinsight中近100个进程的完全讲解。包括它们是什么,它们怎么工作,什么时候应该使用它们以及它们能我们做些什么等。
  \item 解释进程中的每个参数的含义以及调整这些参数的影响,尤其是那些能为我们手头的操作提供切实好处的参数。
  \item 提供在处理工作流中使用特定工具的建议,包括怎么使用这些工具等。
\end{itemize}


\section{这个参考指南是如何构建的}

此参考指南构建的方式非常简单:按字母顺序排列。所以你只用根据首字母顺序去寻找你感兴趣的进程。

每个进程都有一些介绍,有时只有几行,而有时会有洋洋洒洒好几页纸。

何时使用这些进程这部分紧跟在介绍之后,它概述了何时以及为什么我们应该使用这些进程。在这之后还有详细的参数介绍。

尽管此参考指南描述了如何去使用每个进程,也给出了许多推荐的参数,但是所有的进程窗口都只展示了这个进程的默认参数。如果需要有插图的实例,请翻阅完全手册。

该指南的数字版本是一个动态的/交互式的PDF格式。当在一个进程的文档中第一次提到一个另外的进程时,点击它就能跳转到该进程的文档。

大多数进程都是包含介绍的,但是并没有讲得太深。这是因为此参考指南是有意这样设计的。我们可以直接跳转到任何进程并能立马得到我们想要的几乎所有信息,如果过于详细地介绍 Pixinsight 中的所有进程可能会导致重复,因为不同进程间的很多参数都非常相似甚至完全相同。不仅如此,在使用特定的进程时,还有一些概念需要被理解,然而在每个进程中对这些概念相关的解释可能是过量的。因此我需要在重复介绍和介绍不够之间取得平衡。