\chapter{Glossary}


\section{Acronyms}

\begin{longtblr}[
    halign = c,
    caption = {Acronyms},
    label = {tblr:Acronyms},
    note{a} = {第一个表注。},
    remark{注意} = {一些常规说明,一些常规说明,一些常规说明。},
  ]{
    colspec = {X[c]X[c]X[c]}, 
    width = \textwidth,
    rowhead = 1,
    hline{1,Z} = {0.08em, solid},
    hline{2} = {0.05em, solid},
  }
  缩写   & 全称   & 中文 \\
  Alpha   & foo\TblrNote{a} & 中文 \\
  Epsilon & foo & 中文 \\
  Iota    & foo & 中文 \\
  Nu      & foo & 中文 \\
  Rho     & foo & 中文 \\
  Phi     & foo & 中文 \\
  Alpha   & foo & 中文 \\
  Epsilon & foo & 中文 \\
  Iota    & foo & 中文 \\
  Nu      & foo & 中文 \\
  Rho     & foo & 中文 \\
  Phi     & foo & 中文 \\
  Alpha   & foo & 中文 \\
  Epsilon & foo & 中文 \\
  Iota    & foo & 中文 \\
  Nu      & foo & 中文 \\
  Rho     & foo & 中文 \\
  Phi     & foo & 中文 \\
  Alpha   & foo & 中文 \\
  Epsilon & foo & 中文 \\
  Iota    & foo & 中文 \\
  Nu      & foo & 中文 \\
  Rho     & foo & 中文 \\
  Phi     & foo & 中文 \\
  Alpha   & foo & 中文 \\
  Epsilon & foo & 中文 \\
  Iota    & foo & 中文 \\
  Nu      & foo & 中文 \\
  Rho     & foo & 中文 \\
  Phi     & foo & 中文 \\
  foot1    & foo & 中文 \\
\end{longtblr}



\section{Definitions}

\begin{description}
  \item[apple] 一种水果
  \item[egg] 一种吃的 
\end{description}